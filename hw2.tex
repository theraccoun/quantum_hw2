\documentclass[12pt]{article}

\newcommand{\piRsquare}{\pi r^2}

\title{Quant. Comp. HW - 2}
\author{Steven MacCoun}
\date{Oct. 18, 2005}

\begin{document}
\maketitle						% automatic title!



\section{Is Valid Superposition?}

Determine if the state $|\phi>$ is a valid superposition $$|\phi> = \frac{1}{1+i} |0> + \frac{1}{1-i}|1>$$
The state is valid provided that  it is length normalized to one.
$$|\frac{1}{1+i}|^2 + |\frac{1}{1-i}|^2$$ 
$$= \frac{1}{(1+i)(1-i)} + \frac{1}{(1-i)(1+i)}$$
$$= (1/2) + (1/2) = 1$$
Therefore the state is \textbf{valid}.

\section{Find valid superposition}
\section{}
\section{Deutsch Problem}

Suppose we take $U_{f}$ from the Deutsch problem and compute
$$(H\otimes 2)U_{f} (H\otimes 2)|11>$$
Recall that for the Deutsch function:
\\
\begin{table}
    \begin{tabular}{|l|l|l|}
        \hline
        ~  & X=0 & X=1 \\ \hline
        $f_{0}$ & 0   & 0   \\ \hline
        $f_{1}$ & 0   & 1   \\  \hline
        $f_{2}$ & 1   & 0   \\ \hline
        $f_{3}$ & 1   & 1   \\
        \hline
    \end{tabular}
\end{table}
\\
First let's apply the first hadamard gate and calls this intermediate state $|a>$:
$$|a> = (H\otimes H)|11>$$
$$= \frac{1}{2} (|0> - |1>)(|0> - |1>)$$
$$=\frac{1}{2}(|0>|0> - |0>|1> - |1>|0> - |1>|1>)$$
Now let's consider the two cases $f(0)=f(1)$ and $f(0) \neq f(1)$ :
\\\\ Case 1: $f(0)=f(1)$
\\This implies we are using either f0 or f3.
\\For f3:
$$ U_{f_{3}}|a> = U_{f_{3}} [\frac{1}{2}(|0>|0> - |0>|1> - |1>|0> - |1>|1>) ]$$
$$ = (1/2)[|0>|0 \otimes 1> - |0>|1 \otimes 1> - |1>|0 \otimes 1> - |1>|1 \otimes 1>$$
$$ = (1/2)[|0>|1> - |0>|0> - |1>|1> - |1>|0>] $$
Applying the final Hadamard to this:
$$(H\otimes H)\frac{1}{2}[|01> - |00> - |11> - |10>] $$
$$ = \frac{1}{4}[ (|00> + |01> - |10> + |11>) -  (|00> + |01> + |10> + |11>)$$$$- (|00> - |01> - |10> - |11>) - (|00> + |01> - |10> - |11>)]$$
$$ = \frac{1}{4}[-2|00> + 2|11>$$
$$ = \frac{1}{2}[|00> + |11>$$
\\For f0:
$$ U_{f_{0}}|a> = U_{f_{0}} [\frac{1}{2}(|0>|0> - |0>|1> - |1>|0> - |1>|1>) ]$$
Since $U_{f_{0}}$ always returns 0, the computation is straightforward since anything XOR'ed with 0 is just itself
$$ = (1/2)[|0>|0> - |0>|1> - |1>|0> - |1>|1>$$


\subsection{Numbered formulae}

Useful Hadamards:
\begin{equation}
	(H\otimes H) |00> = \frac{1}{2}(|00> + |01> + |10> + |11>)
\end{equation}
\begin{equation}
	(H\otimes H) |01> = \frac{1}{2}(|00> - |01> + |10> - |11>)
\end{equation}
\begin{equation}
	(H\otimes H) |10> = \frac{1}{2}(|00> + |01> - |10> - |11>)
\end{equation}
\begin{equation}
	(H\otimes H) |11> = \frac{1}{2}(|00> - |01> - |10> - |11>)
\end{equation}

Use the \emph{equation} environment to get numbered formulae, e.g.,
\begin{equation}
	y_{i+1} = x_{i}^{2n} - \sqrt{5}x_{i-1}^{n} + \sqrt{x_{i-2}^7} -1
\end{equation}

\begin{equation}
	\frac{\partial u}{\partial t} + \nabla^{4}u + \nabla^{2}u +
        \frac12    |\nabla u|^{2}~ =~ c^2
\end{equation}

\section{Acknowledgments}

Thanks to my buddies {\AE}schyulus and Chlo\"{e},
who helped me define the macro \verb9\piRsquare9
which is $\piRsquare$.
The end.

\end{document} 