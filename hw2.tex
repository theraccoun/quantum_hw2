\documentclass[12pt]{article}
\usepackage{amsmath}
\newcommand{\piRsquare}{\pi r^2}

\title{Quant. Comp. HW - 2}
\author{Steven MacCoun}
\date{Oct. 18, 2005}

\begin{document}
\maketitle						% automatic title!



\section{Is Valid Superposition?}

Determine if the state $|\phi>$ is a valid superposition $$|\phi> = \frac{1}{1+i} |0> + \frac{1}{1-i}|1>$$
The state is valid provided that  it is length normalized to one.
$$|\frac{1}{1+i}|^2 + |\frac{1}{1-i}|^2$$ 
$$= \frac{1}{(1+i)(1-i)} + \frac{1}{(1-i)(1+i)}$$
$$= (1/2) + (1/2) = 1$$
Therefore the state is \textbf{valid}.

\section{Find valid superposition}

Given $|\phi> = (1/2)|00> + \frac{x}{2 \sqrt{2}}|01> + \frac{1}{2 \sqrt{2}}|10> +  \frac{1}{2}|11>$, what values of x would make this a valid superposition?

Again the normalization condition is applied:
$$|\frac{1}{2}|^{2} + |\frac{x}{2 \sqrt{2}}|^2 + |\frac{1}{2 \sqrt{2}}|^2 +  |\frac{1}{2}|^2 = 1$$
$$(1/4) + \frac{|x|^2}{8} + (1/8) + (1/4) = 1$$
$$\frac{(5 + |x|^2)}{8} = 1$$
$$|x|^2 = 3$$
Since x can be imaginary, all we know is that the real part must be equal to $\sqrt{3}$
Therefore:
\[
\boxed{x = \sqrt{3} + bi}
\]
\section{}

Let $|\psi> = \frac{1}{\sqrt{3}}(|00> + |01> + |10>)$
Find:
$$|\phi> = (H \otimes H)|\psi>$$
$$ =  \frac{1}{\sqrt{6}}[(|00> + |01> + |10> + |11>)$$$$ + (|00> - |01> + |10> - |11>) $$$$+ (|00> + |01> - |10> + |11>)]$$
$$ =  \frac{1}{\sqrt{6}}[3|00>+ |01> + |10> + |11>]$$
\[
\boxed{ = \frac{1}{2}|00> + \frac{1}{6}|1> + \frac{1}{6}|2> - \frac{1}{6}|3>}
\]
\section{Deutsch Problem}

Suppose we take $U_{f}$ from the Deutsch problem and compute
$$(H\otimes 2)U_{f} (H\otimes 2)|11>$$
Recall that for the Deutsch function:
\\
\begin{table}[h]
\centering
    \begin{tabular}{|l|l|l|}
        \hline
        ~  & X=0 & X=1 \\ \hline
        $f_{0}$ & 0   & 0   \\ \hline
        $f_{1}$ & 0   & 1   \\  \hline
        $f_{2}$ & 1   & 0   \\ \hline
        $f_{3}$ & 1   & 1   \\
        \hline
    \end{tabular}
\end{table}
\\
First let's apply the first hadamard gate and calls this intermediate state $|a>$:
$$|a> = (H\otimes H)|11>$$
$$= \frac{1}{2} (|0> - |1>)(|0> - |1>)$$
$$=\frac{1}{2}(|0>|0> - |0>|1> - |1>|0> - |1>|1>)$$
In computing the Hadamard on this next, a general equation can be set up for the four 'f' functions. Since anything XOR'ed with itself is 0, and anything XOR'ed with 1 is it's complement, we can write $U_{f}|a>$ as:
$$U_{f}|a> = \frac{1}{2}[|0>|f(0)> - |0>|\bar{f}(0)> - |1>|f(1)> - |1>|\bar{f}(1)>] $$
Now let's consider the two cases $f(0)=f(1)$ and $f(0) \neq f(1)$ :
\\\\ Case 1: $f(0)=f(1)$
\\This implies we are using either f0 or f3.
\\\\$f_{0}$:
$$ HU_{f_{0}}|a> = H \frac{1}{2}[|00> - |01> - |10> - |11>]$$
$$  = \frac{1}{2}[-|00> + |11> + |11> + |01> + |10>]$$
\[
\boxed{ = \frac{1}{2}[-|00> + 2|11> + |01> + |10>] \>\>\>\>\>\>(f_{0}) }
\] 
\\$f_{3}$:
$$ HU_{f_{3}}|a> = H  \frac{1}{2}[|01> - |00> - |11> - |10>]$$
$$ = \frac{1}{4}[ (|00> + |01> - |10> + |11>) -  (|00> + |01> + |10> + |11>)$$$$- (|00> - |01> - |10> - |11>) - (|00> + |01> - |10> - |11>)]$$
$$ = \frac{1}{4}[-2|00> + 2|11>$$
\[
\boxed{ = \frac{1}{2}[|00> + |11> 	\>\>\>\>\>\>(f_{3})}
\]
\\\\
Now for the case of $f(0) \neq f(1)$, so we look at f1 and f2. Note that applying U is easy here too, because we are always XOR a bit with it's complement, which always yields 1. Since the input gates will be the same for $f_{1} and f_{2}$, then each should return the same state:
\\\\
$f_{1}:$
$$ HU_{f_1}|a> = H \frac{1}{2}[|00> - |01> - |11> - |10>]  $$
$$ = \frac{1}{4}[2|01> + 2|11> - 2|00> + 2|10> + 2|11>] $$
\[
\boxed{ = \frac{1}{2}[-|00> + |01> + |10> + |11>] \>\>\>\> (f_{1}) }
\]
$f_{2}:$
$$ HU_{f_3} |a> = H \frac{1}{2}[|01> - |00> - |11> - |10>]$$
$$ = \frac{1}{4}[-2|01> - 2|11> - 2|00> + 2|10> + 2|11> $$
\[
\boxed{ = \frac{1}{2}[-|00> + |01> + |10>] \>\>\>\>(f_{2})}
\]\

\section{Bernstein-Verizani}

Take f(x) from the Bernstein-Verizani problem and compute:
$$U_{f}(H^{n \otimes 1})(|0>_{n}|1>_{1})$$
$$ = U_{f} (\frac{1}{2^{n/2}}\displaystyle\sum\limits_{x=0}^{2^{n} - 1} |x>) \frac{1}{\sqrt{2}}(|0> - |1>)$$
Since $ U_{f}|x>_{n} \frac{1}{\sqrt{2}}(|0>-|1>) = (-1)^{f(x)}|x>_{n} \frac{1}{\sqrt{2}}(|0> - |1>) $
$$ = \frac{1}{2^{n/2}} (\displaystyle\sum\limits_{x=0}^{2^{n} - 1} (-1)^{f(x)}|x>) \frac{1}{\sqrt{2}}(|0> - |1>) $$
\[
\boxed{ = \frac{1}{2^{(n+1)/2}} (\displaystyle\sum\limits_{x=0}^{2^{n} - 1} (-1)^{f(x)}|x>)(|0> - |1>)}
\]
Recall that the function f(x) in the Bernstein-Verizani is $ a \cdot x$. Therefore, f(x) is always either 0 or 1, and so each value in the sum is either $|x>$ or $-|x>$.
The possible values of the input register are:
\[
\boxed{\frac{1}{2^{n/2}} (\displaystyle\sum\limits_{x=0}^{2^{n} - 1} (-1)^{f(x)}|x>)}
\]
The only possible value of the output register is the superposed state:
\[
\boxed{ \frac{1}{2^{(n+1)/2}}(|0> - |1>)}
\]

\subsection{Numbered formulae}

Useful Hadamards:
\begin{equation}
	(H\otimes H) |00> = \frac{1}{2}(|00> + |01> + |10> + |11>)
\end{equation}
\begin{equation}
	(H\otimes H) |01> = \frac{1}{2}(|00> - |01> + |10> - |11>)
\end{equation}
\begin{equation}
	(H\otimes H) |10> = \frac{1}{2}(|00> + |01> - |10> - |11>)
\end{equation}
\begin{equation}
	(H\otimes H) |11> = \frac{1}{2}(|00> - |01> - |10> - |11>)
\end{equation}

Use the \emph{equation} environment to get numbered formulae, e.g.,
\begin{equation}
	y_{i+1} = x_{i}^{2n} - \sqrt{5}x_{i-1}^{n} + \sqrt{x_{i-2}^7} -1
\end{equation}

\begin{equation}
	\frac{\partial u}{\partial t} + \nabla^{4}u + \nabla^{2}u +
        \frac12    |\nabla u|^{2}~ =~ c^2
\end{equation}

\section{Acknowledgments}

Thanks to my buddies {\AE}schyulus and Chlo\"{e},
who helped me define the macro \verb9\piRsquare9
which is $\piRsquare$.
The end.

\end{document} 